\documentclass{beamer}
\usepackage{amsmath,amssymb,longtable,hhline}

% \usepackage{minted}
% \usemintedstyle{tango} % bw
\usetheme{Madrid}
\usecolortheme{beaver}

\RequirePackage{tabularx}
\RequirePackage{xcolor}
\usepackage{pifont}
\definecolor{darkding}{RGB}{200,56,0}
\setbeamertemplate{itemize item}{\scriptsize\hbox{\color{darkding}{\bfseries\ding{113}}}}


\usepackage{polyglossia}
\setmainlanguage{russian}
\setotherlanguage{english}
\setkeys{russian}{babelshorthands=true}

\setmainfont{Linux Biolinum O}
\setromanfont{Linux Biolinum O}
\setsansfont{Linux Biolinum O}
\setmonofont{Fira Code}

\newfontfamily{\cyrillicfont}{Linux Biolinum O}
\newfontfamily{\cyrillicfontrm}{Linux Biolinum O}
\newfontfamily{\cyrillicfonttt}{Fira Code}
\newfontfamily{\cyrillicfontsf}{Linux Biolinum O}
\newcommand{\HHuge}{\fontsize{60}{60}\selectfont}


\RequirePackage{luatextra}
\RequirePackage{graphicx}
\graphicspath{{pics/}}

\defaultfontfeatures{Ligatures={TeX,Required},Scale=MatchLowercase} % - -- --- ffi

\setmainfont{Fira Mono}

\begin{document}

\title[Учет радиодеталей]{Программная система учета радиодеталей}
\subtitle{Этап проектирования}
\author[Е.~А.~Черкашин]{Евгений Александрович Черкашин, к.т.н., доцент каф.~ИС}
\date{Иркутск -- 2021}
\institute[ИДСТУ СО РАН]{Институт динамики систем и теории управления Сибирского отделения ....}
% \logo{\includegraphics[width=2cm]{fig1.pdf}}

\frame{\titlepage}

\begin{frame}
  \vfill
  \begin{block}{}
    \fontsize{20}{25}\selectfont \bfseries\centering
    \alert{Программная система учета радиодеталей}
  \end{block}
\vspace{3em}
\noindent\begin{tabularx}{\textwidth} {
  >{\raggedright\arraybackslash}X
  >{\raggedright}X }
&
Студентки 4 курса группы 2361\\
Фамилия Имя Отчество\\[0.3em]
% Направление\;: 02.03.03~--~Математическое обеспечение и администрирование информационных систем\\[2em]
Руководитель:\\
канд.~техн.~наук доцент\\
Иванов Иван Иванович
% Курсовая работа защищена с оценкой\\[1em] \underline{\hspace{3cm}}
\end{tabularx}
\vfill
\begin{center}
  Иркутск -- 2021
\end{center}%
\end{frame}

\begin{frame}[fragile]{Объект автоматизации}

  \textbf{Объектом} автоматизации выступал

  \textbf{Предмет} автоматизации -- это \ldots

\end{frame}

\begin{frame}
  \frametitle{Другой способ выделить содержимое}

  \begin{block}{Объект автоматизации}
    Процесс проектирования печатных плат
  \end{block}

  \begin{alertblock}{Предмет автоматизации}
    Ведение базы данных радиодеталей со ссылками на документациюr
  \end{alertblock}
\end{frame}

\begin{frame}{Пример приложения представленного решения}
  \begin{columns}
    \begin{column}{0.5\linewidth}
      \begin{block}{Свойства результата}
        \begin{itemize}
        \item Результат состоит из трех компонент
        \item Одна из компонент -- это текст
        \end{itemize}
      \end{block}
    \end{column}
    \begin{column}{0.45\linewidth}
      \includegraphics[width=0.9\linewidth]{fig1.pdf}
    \end{column}
  \end{columns}

\end{frame}


\begin{frame}
  \vfill
  \centering
  \Large\bfseries Вопросы?
  \vfill
\end{frame}

\end{document}




%%% Local Variables:
%%% mode: latex
%%% TeX-master: t
%%% End:
